\documentclass[18pt]{beamer}
\usepackage{etex}
\usepackage{beamerthemesplit} % new
\usetheme[secheader]{Madrid}
%\usetheme{Warsaw}
\usecolortheme{dolphin}
\usepackage{graphicx}
%\usepackage{landscape}
\usepackage{amsmath}
\usepackage{amsfonts}
\usepackage{amssymb}
\usepackage[frenchb]{babel}
\usepackage{amstext}
\usepackage[utf8]{inputenc}
%\usepackage[T1]{fontenc}
\usepackage{pgf,tikz}
\usepackage{listings}


%\usepackage{titletoc}

\newcounter{moncompteur}
\newtheorem{q}[moncompteur]{\textbf{Question}}{}
\newtheorem{prop}[moncompteur]{\textbf{Proposition}}{}
\newtheorem{df}[moncompteur]{\textbf{Définition}}{}
\newtheorem*{df*}{\textbf{Définition}}{}
\newtheorem*{prop*}{\textbf{Proposition}}{}
\newtheorem*{theo*}{\textbf{Théorème}}{}

\newtheorem{rem}[moncompteur]{\textbf{Remarque}}{}
\newtheorem{theo}[moncompteur]{\textbf{Théorème}}{}
\newtheorem{conj}[moncompteur]{\textbf{Conjecture}}{}
\newtheorem{cor}[moncompteur]{\textbf{Corollaire}}{}
\newtheorem{lm}[moncompteur]{\textbf{Lemme}}{}
%\newtheorem{nota}[moncompteur]{\textbf{Notation}}{}
%\newtheorem{conv}[moncompteur]{\textbf{Convention}}{}
\newtheorem{exa}[moncompteur]{\textbf{Exemple}}{}
\newtheorem{ex}[moncompteur]{\textbf{Exercice}}{}
%\newtheorem{app}[moncompteur]{\textbf{Application}}{}
%\newtheorem{prog}[moncompteur]{\textbf{Algorithme}}{}
%\newtheorem{hyp}[moncompteur]{\textbf{Hypothèse}}{}
\newenvironment{dem}{\noindent\textbf{Preuve}\\}{\flushright$\blacksquare$\\}
\newcommand{\cg}{[\kern-0.15em [}
\newcommand{\cd}{]\kern-0.15em]}
\newcommand{\R}{\mathbb{R}}
\newcommand{\K}{\mathbb{K}}
\newcommand{\N}{\mathbb{N}}
\newcommand{\Z}{\mathbb{Z}}
\newcommand{\C}{\mathbb{C}}
\newcommand{\U}{\mathbb{U}}
\newcommand{\Q}{\mathbb{Q}}
\newcommand{\B}{\mathbb{B}}
\newcommand{\Prob}{\mathbb P}
\newcommand{\card}{\mathrm{card}}
\newcommand{\norm}[1]{\left\lVert#1\right\rVert}
%\pgfplotsset{compat=newest}
\newcommand{\La}{\mathcal{L}}
\newcommand{\Ne}{\mathcal{N}}
\newcommand{\D}{\mathcal{D}}
\newcommand{\Ss}{\textsc{safestay}}
\newcommand{\Sg}{\textsc{safego}}
\newcommand{\M}{\textsc{move}}
\newcommand{\E}{\mathcal{E}}
\newcommand{\V}{\mathcal V}
\newcommand{\A}{\mathcal A}
\newcommand{\T}{\mathcal T}
\newcommand{\Ca}{\mathcal C}
\setlength{\parindent}{0pt}
\newcommand{\myrightleftarrows}[1]{\mathrel{\substack{\xrightarrow{#1} \\[-.6ex] \xleftarrow{#1}}}}
\newcommand{\longrightleftarrows}{\myrightleftarrows{\rule{1cm}{0cm}}}
\newcommand{\ZnZ}{\Z/n\Z}


\graphicspath{{./}}


\setbeamertemplate{navigation symbols}{%
	%\insertslidenavigationsymbol
	%\insertframenavigationsymbol
	%\insertsubsectionnavigationsymbol
	%\insertsectionnavigationsymbol
	%\insertdocnavigationsymbol
	%\insertbackfindforwardnavigationsymbol
}
\title{Contextualization of Facebook documents}
\author{Diane Gallois-Wong, Raphaël Rieu-Helft}
\date{February 22, 2016}
\begin{document}

\begin{frame}
	\titlepage
\end{frame}

\begin{frame}
	\frametitle{The Neemi platform}
\begin{itemize}
	\item{Django webapp that extracts user data from various social networks and stores it in a MongoDB database}
	\vspace{1em}
	\item{Our project is a fork of Neemi with added features}
\end{itemize}
\end{frame}

\begin{frame}
	\frametitle{Contextualizing a Facebook document}
	\begin{itemize}
		\item{Choose a photo or event in the database}
		\item{Build a corresponding RDF graph}
		\item{Fetch related items in the database}
		\item{Build their RDF graphs and absorb them into the first one}
	\end{itemize}
\end{frame}
\begin{frame}
	\frametitle{Facebook API limitations}
	\begin{itemize}
		\item{The latest versions of the API is very restrictive with data from friends who didn't install our app}
		\item{Getting interesting data is difficult}
		\item{New Neemi feature : add mock documents}
	\end{itemize}
\end{frame}
\begin{frame}
	\frametitle{Absorbing graphs : algorithm}
	
	%TODO
	
\end{frame}
\begin{frame}
	\frametitle{Some technical choices}
	\begin{itemize}
		\item{Merging location information : geocode everything, then compare coordinates}
		\item{Deciding if a document is relevant to the event : time-based for now, could be improved}
	\end{itemize}
\end{frame}
\begin{frame}
	\frametitle{Potential extensions}
	\begin{itemize}
		\item{Handle documents from other sources than Facebook : new methods need to be added but no conceptual difficulty}
		\item{Better heuristics to deduce if an item is related to the event}
	\end{itemize}
\end{frame}

\end{document}